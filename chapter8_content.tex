% Chapter 8 Content - Functions
% This file contains the content for Chapter 8 based on actual Python files

This document contains the essential keywords and definitions from Chapter 8 of "Python Crash Course" along with their corresponding code examples.

\subsection*{1. Function - Reusable Code Block}
\textbf{Definition}: A named block of code that performs a specific task and can be called from other parts of your program.

\lstinputlisting[language=Python, caption=Basic function definition]{Chapter08/801_greeter.py}

\subsection*{2. def Statement - Function Definition}
\textbf{Definition}: A statement that defines a function, specifying its name and parameters.

\begin{lstlisting}
def greet_user():
    """Display a simple greeting."""
    print("Hello!")
\end{lstlisting}

\subsection*{3. Parameter - Function Input}
\textbf{Definition}: A piece of information that a function needs to do its job, specified in the function definition.

\begin{lstlisting}
def greet_user(username):
    print(f"Hello, {username.title()}!")
\end{lstlisting}

\subsection*{4. Argument - Function Call Value}
\textbf{Definition}: A piece of information that's passed from a function call to a function.

\begin{lstlisting}
greet_user('jesse')  # 'jesse' is the argument
\end{lstlisting}

\subsection*{5. Return Value - Function Output}
\textbf{Definition}: The value that a function returns to the calling line of code.

\lstinputlisting[language=Python, caption=Function with return value]{Chapter08/803_formatted_name.py}

\subsection*{6. Default Parameter Value - Optional Arguments}
\textbf{Definition}: A parameter that has a default value, making it optional when calling the function.

\begin{lstlisting}
def get_formatted_name(first_name, last_name, middle_name=''):
    if middle_name:
        full_name = f"{first_name} {middle_name} {last_name}"
    else:
        full_name = f"{first_name} {last_name}"
    return full_name.title()
\end{lstlisting}

\subsection*{7. Positional Arguments - Order-Based}
\textbf{Definition}: Arguments that must be passed to a function in the same order as the parameters are defined.

\begin{lstlisting}
def describe_pet(animal_type, pet_name):
    print(f"\nI have a {animal_type}.")
    print(f"My {animal_type}'s name is {pet_name.title()}.")

describe_pet('hamster', 'harry')
\end{lstlisting}

\subsection*{8. Keyword Arguments - Name-Based}
\textbf{Definition}: Arguments that are passed to a function by parameter name, allowing any order.

\begin{lstlisting}
describe_pet(animal_type='hamster', pet_name='harry')
describe_pet(pet_name='harry', animal_type='hamster')
\end{lstlisting}

\subsection*{9. Arbitrary Arguments - *args}
\textbf{Definition}: A parameter that allows a function to accept any number of arguments.

\lstinputlisting[language=Python, caption=Arbitrary arguments]{Chapter08/807_pizza.py}

\subsection*{10. Arbitrary Keyword Arguments - **kwargs}
\textbf{Definition}: A parameter that allows a function to accept any number of keyword arguments.

\begin{lstlisting}
def build_profile(first, last, **user_info):
    """Build a dictionary containing everything we know about a user."""
    user_info['first_name'] = first
    user_info['last_name'] = last
    return user_info

user_profile = build_profile('albert', 'einstein',
                           location='princeton',
                           field='physics')
\end{lstlisting}

\subsection*{11. Docstring - Function Documentation}
\textbf{Definition}: A string that describes what a function does, enclosed in triple quotes.

\begin{lstlisting}
def greet_user(username):
    """Display a simple greeting."""
    print(f"Hello, {username.title()}!")
\end{lstlisting}

\subsection*{12. Module - Code Organization}
\textbf{Definition}: A file containing functions and variables that can be imported into other programs.

\begin{lstlisting}
# pizza.py
def make_pizza(size, *toppings):
    """Summarize the pizza we are about to make."""
    print(f"\nMaking a {size}-inch pizza with the following toppings:")
    for topping in toppings:
        print(f"- {topping}")
\end{lstlisting}

\subsection*{13. import Statement - Module Usage}
\textbf{Definition}: A statement that makes functions and variables from a module available in your program.

\begin{lstlisting}
import pizza
pizza.make_pizza(16, 'pepperoni')
pizza.make_pizza(12, 'mushrooms', 'green peppers', 'extra cheese')
\end{lstlisting}

\subsection*{14. from...import Statement - Selective Import}
\textbf{Definition}: A statement that imports specific functions from a module.

\begin{lstlisting}
from pizza import make_pizza
make_pizza(16, 'pepperoni')
\end{lstlisting}

\section*{Practical Examples from Chapter 8}

\subsection*{Working with Functions}
Chapter 8 introduces functions and their various forms. Here are the key files:

\textbf{Basic Function Definition:}
\lstinputlisting[language=Python, caption=Chapter08/801\_greeter.py]{Chapter08/801_greeter.py}

\textbf{Functions with Return Values:}
\lstinputlisting[language=Python, caption=Chapter08/803\_formatted\_name.py]{Chapter08/803_formatted_name.py}

\textbf{Functions with Arbitrary Arguments:}
\lstinputlisting[language=Python, caption=Chapter08/807\_pizza.py]{Chapter08/807_pizza.py}

\textbf{To run these programs:}
\begin{verbatim}
python Chapter08/801_greeter.py
python Chapter08/803_formatted_name.py
python Chapter08/807_pizza.py
\end{verbatim}

\section*{Summary}
Chapter 8 focuses on functions, which are blocks of code that perform specific tasks. You learn how to define functions, pass information to them, and get information back from them.

\section*{Key Takeaways}
\begin{itemize}
    \item Functions are reusable blocks of code
    \item Use def to define a function
    \item Parameters receive information in functions
    \item Arguments provide information to functions
    \item return sends a value back to the calling line
    \item Default parameters make arguments optional
    \item Positional arguments must be in correct order
    \item Keyword arguments can be in any order
    \item *args accepts any number of arguments
    \item **kwargs accepts any number of keyword arguments
    \item Docstrings document what functions do
    \item Modules organize code into files
    \item import makes modules available
    \item from...import brings specific functions
\end{itemize} 