% Chapter 5 Content - if Statements
% This file contains the content for Chapter 5 based on actual Python files

This document contains the essential keywords and definitions from Chapter 5 of "Python Crash Course" along with their corresponding code examples.

\subsection*{1. if Statement - Conditional Execution}
\textbf{Definition}: A statement that allows you to examine the current state of a program and respond appropriately.

\lstinputlisting[language=Python, caption=Basic if statements]{Chapter05/501_cars.py}

\subsection*{2. Conditional Test - True/False Check}
\textbf{Definition}: An expression that can be evaluated as True or False, used to decide whether code should be executed.

\begin{lstlisting}
car = 'bmw'
car == 'bmw'  # True
car == 'audi'  # False
\end{lstlisting}

\subsection*{3. Equality Operator - ==}
\textbf{Definition}: An operator that checks if two values are equal, returning True or False.

\begin{lstlisting}
answer = 42
if answer == 42:
    print("Correct!")
\end{lstlisting}

\subsection*{4. Inequality Operator - !=}
\textbf{Definition}: An operator that checks if two values are not equal, returning True or False.

\lstinputlisting[language=Python, caption=Inequality testing]{Chapter05/503_magic_number.py}

\subsection*{5. elif Statement - Multiple Conditions}
\textbf{Definition}: A statement that allows you to check multiple conditions when the first if statement is False.

\begin{lstlisting}
age = 12
if age < 4:
    price = 0
elif age < 18:
    price = 5
else:
    price = 10
\end{lstlisting}

\subsection*{6. else Statement - Default Action}
\textbf{Definition}: A statement that provides a default action when all previous conditions are False.

\begin{lstlisting}
if age < 4:
    price = 0
else:
    price = 10
\end{lstlisting}

\subsection*{7. in Operator - Membership Test}
\textbf{Definition}: An operator that checks if a value exists in a list or other collection.

\lstinputlisting[language=Python, caption=Membership testing]{Chapter05/502_toppings.py}

\subsection*{8. Boolean Values - True/False}
\textbf{Definition}: Values that represent the truth or falsity of a condition.

\begin{lstlisting}
game_active = True
can_edit = False
\end{lstlisting}

\subsection*{9. and Operator - Multiple Conditions}
\textbf{Definition}: An operator that returns True only if all conditions are True.

\begin{lstlisting}
age_0 = 22
age_1 = 18
age_0 >= 21 and age_1 >= 21  # False
\end{lstlisting}

\subsection*{10. or Operator - Alternative Conditions}
\textbf{Definition}: An operator that returns True if any condition is True.

\begin{lstlisting}
age_0 = 22
age_1 = 18
age_0 >= 21 or age_1 >= 21  # True
\end{lstlisting}

\subsection*{11. not Operator - Negation}
\textbf{Definition}: An operator that negates a condition, returning the opposite boolean value.

\begin{lstlisting}
banned_users = ['andrew', 'carolina', 'david']
user = 'marie'
if user not in banned_users:
    print(f"{user.title()}, you can post a response if you wish.")
\end{lstlisting}

\section*{Practical Examples from Chapter 5}

\subsection*{Working with Conditional Statements}
Chapter 5 introduces if statements and conditional logic. Here are the key files:

\textbf{Basic if Statements:}
\lstinputlisting[language=Python, caption=Chapter05/501\_cars.py]{Chapter05/501_cars.py}

\textbf{Complex Conditional Logic:}
\lstinputlisting[language=Python, caption=Chapter05/502\_toppings.py]{Chapter05/502_toppings.py}

\textbf{Numerical Comparisons:}
\lstinputlisting[language=Python, caption=Chapter05/503\_magic\_number.py]{Chapter05/503_magic_number.py}

\textbf{To run these programs:}
\begin{verbatim}
python Chapter05/501_cars.py
python Chapter05/502_toppings.py
python Chapter05/503_magic_number.py
\end{verbatim}

\section*{Summary}
Chapter 5 focuses on if statements and conditional logic. You learn how to make decisions in your programs, test conditions, and execute different code based on the results of those tests.

\section*{Key Takeaways}
\begin{itemize}
    \item if statements allow programs to make decisions
    \item Use == to test for equality, != for inequality
    \item elif provides additional conditions to test
    \item else provides a default action
    \item Use in to test if a value is in a list
    \item and requires all conditions to be True
    \item or requires at least one condition to be True
    \item not negates a condition
    \item Boolean values are True and False
    \item Conditional tests can be simple or complex
\end{itemize} 