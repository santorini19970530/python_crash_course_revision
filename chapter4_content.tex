% Chapter 4 Content - Working with Lists
% This file contains the content for Chapter 4 based on actual Python files

This document contains the essential keywords and definitions from Chapter 4 of "Python Crash Course" along with their corresponding code examples.

\subsection*{1. for Loop - Iterating Through Lists}
\textbf{Definition}: A loop that runs once for each item in a list or other collection.

\lstinputlisting[language=Python, caption=Basic for loop]{Chapter04/401_magicians.py}

\subsection*{2. Loop Variable - Current Item}
\textbf{Definition}: The variable that holds the current item being processed in a loop.

\begin{lstlisting}
for magician in magicians:
    print(magician)  # magician is the loop variable
\end{lstlisting}

\subsection*{3. Indentation - Code Blocking}
\textbf{Definition}: The use of spaces or tabs to indicate which lines of code belong together in a block.

\begin{lstlisting}
for magician in magicians:
    print(magician)  # This line is indented
    print("Great trick!")  # This line is also indented
print("Thank you!")  # This line is not indented
\end{lstlisting}

\subsection*{4. range() Function - Number Sequences}
\textbf{Definition}: A function that generates a sequence of numbers for use in loops.

\lstinputlisting[language=Python, caption=Using range()]{Chapter04/402_first_numbers.py}

\subsection*{5. List Comprehension - Compact Lists}
\textbf{Definition}: A way to create lists using a compact syntax with loops and conditions.

\lstinputlisting[language=Python, caption=List comprehensions]{Chapter04/404_squares.py}

\subsection*{6. Slicing - List Portions}
\textbf{Definition}: A way to work with a portion of a list by specifying start and end indices.

\begin{lstlisting}
players = ['charles', 'martina', 'michael', 'florence', 'eli']
print(players[0:3])  # ['charles', 'martina', 'michael']
print(players[1:4])  # ['martina', 'michael', 'florence']
print(players[:4])   # ['charles', 'martina', 'michael', 'florence']
print(players[2:])   # ['michael', 'florence', 'eli']
\end{lstlisting}

\subsection*{7. Copying Lists - Creating Duplicates}
\textbf{Definition}: Creating a copy of a list to avoid modifying the original.

\begin{lstlisting}
my_foods = ['pizza', 'falafel', 'carrot cake']
friend_foods = my_foods[:]  # Create a copy
\end{lstlisting}

\section*{Practical Examples from Chapter 4}

\subsection*{Working with Lists and Loops}
Chapter 4 introduces loops and advanced list operations. Here are the key files:

\textbf{Basic Loops:}
\lstinputlisting[language=Python, caption=Chapter04/401\_magicians.py]{Chapter04/401_magicians.py}

\textbf{Number Sequences:}
\lstinputlisting[language=Python, caption=Chapter04/402\_first\_numbers.py]{Chapter04/402_first_numbers.py}

\textbf{List Comprehensions:}
\lstinputlisting[language=Python, caption=Chapter04/404\_squares.py]{Chapter04/404_squares.py}

\textbf{To run these programs:}
\begin{verbatim}
python Chapter04/401_magicians.py
python Chapter04/402_first_numbers.py
python Chapter04/404_squares.py
\end{verbatim}

\section*{Summary}
Chapter 4 focuses on working with lists using loops. You learn how to iterate through lists, use the range() function, and create list comprehensions for more efficient code.

\section*{Key Takeaways}
\begin{itemize}
    \item for loops iterate through each item in a list
    \item Use proper indentation to define loop blocks
    \item range() generates sequences of numbers
    \item List comprehensions create lists efficiently
    \item Slicing allows you to work with portions of lists
    \item Copy lists to avoid modifying originals
\end{itemize} 