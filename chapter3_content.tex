% Chapter 3 Content - Introducing Lists
% This file contains the content for Chapter 3 based on actual Python files

This document contains the essential keywords and definitions from Chapter 3 of "Python Crash Course" along with their corresponding code examples.

\subsection*{1. List - Collection of Items}
\textbf{Definition}: A collection of items in a particular order, enclosed in square brackets and separated by commas.

\lstinputlisting[language=Python, caption=Basic list operations]{Chapter03/301_bicycles.py}

\subsection*{2. Index - Position in List}
\textbf{Definition}: The position of an item in a list, starting from 0 for the first item.

\begin{lstlisting}
bicycles = ['trek', 'cannondale', 'redline', 'specialized']
print(bicycles[0])  # trek
print(bicycles[1])  # cannondale
\end{lstlisting}

\subsection*{3. Negative Index - Accessing from End}
\textbf{Definition}: Using negative numbers to access items from the end of a list (-1 is the last item).

\begin{lstlisting}
bicycles = ['trek', 'cannondale', 'redline', 'specialized']
print(bicycles[-1])  # specialized
print(bicycles[-2])  # redline
\end{lstlisting}

\subsection*{4. Modifying List Elements}
\textbf{Definition}: Changing the value of an item in a list by using its index.

\lstinputlisting[language=Python, caption=Modifying and manipulating lists]{Chapter03/302_motocycles.py}

\subsection*{5. append() Method - Adding to End}
\textbf{Definition}: A method that adds an item to the end of a list.

\begin{lstlisting}
motorcycles = ['honda', 'yamaha', 'suzuki']
motorcycles.append('ducati')
print(motorcycles)  # ['honda', 'yamaha', 'suzuki', 'ducati']
\end{lstlisting}

\subsection*{6. insert() Method - Adding at Position}
\textbf{Definition}: A method that adds an item at a specific position in a list.

\begin{lstlisting}
motorcycles = ['honda', 'yamaha', 'suzuki']
motorcycles.insert(0, 'ducati')
print(motorcycles)  # ['ducati', 'honda', 'yamaha', 'suzuki']
\end{lstlisting}

\subsection*{7. del Statement - Removing by Index}
\textbf{Definition}: A statement that removes an item from a list using its index.

\begin{lstlisting}
motorcycles = ['honda', 'yamaha', 'suzuki']
del motorcycles[0]
print(motorcycles)  # ['yamaha', 'suzuki']
\end{lstlisting}

\subsection*{8. pop() Method - Removing and Returning}
\textbf{Definition}: A method that removes the last item from a list and returns it.

\begin{lstlisting}
motorcycles = ['honda', 'yamaha', 'suzuki']
popped_motorcycle = motorcycles.pop()
print(popped_motorcycle)  # suzuki
print(motorcycles)  # ['honda', 'yamaha']
\end{lstlisting}

\subsection*{9. remove() Method - Removing by Value}
\textbf{Definition}: A method that removes an item from a list by its value.

\begin{lstlisting}
motorcycles = ['honda', 'yamaha', 'suzuki', 'ducati']
motorcycles.remove('ducati')
print(motorcycles)  # ['honda', 'yamaha', 'suzuki']
\end{lstlisting}

\subsection*{10. Empty List - Starting Fresh}
\textbf{Definition}: A list with no items, created using empty square brackets.

\begin{lstlisting}
motorcycles = []
motorcycles.append('honda')
motorcycles.append('yamaha')
print(motorcycles)  # ['honda', 'yamaha']
\end{lstlisting}

\section*{Practical Examples from Chapter 3}

\subsection*{Working with Lists}
Chapter 3 introduces lists and their basic operations. Here are the key files:

\textbf{Basic List Operations:}
\lstinputlisting[language=Python, caption=Chapter03/301\_bicycles.py]{Chapter03/301_bicycles.py}

\textbf{List Modifications:}
\lstinputlisting[language=Python, caption=Chapter03/302\_motocycles.py]{Chapter03/302_motocycles.py}

\textbf{To run these programs:}
\begin{verbatim}
python Chapter03/301_bicycles.py
python Chapter03/302_motocycles.py
\end{verbatim}

\section*{Summary}
Chapter 3 focuses on lists, which are collections of items in a particular order. You learn how to create lists, access items by their position, and modify lists by adding, inserting, and removing items.

\section*{Key Takeaways}
\begin{itemize}
    \item Lists store collections of items in a specific order
    \item Use square brackets to create lists
    \item Access items using their index (starting from 0)
    \item Use negative indices to access items from the end
    \item append() adds items to the end of a list
    \item insert() adds items at a specific position
    \item del removes items by index
    \item pop() removes and returns the last item
    \item remove() removes items by their value
    \item Lists can be modified after creation
\end{itemize} 