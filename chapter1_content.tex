% Chapter 1 Content - Importable version
% This file contains the content for Chapter 1 without the document structure

This document contains the essential keywords and definitions from Chapter 1 of "Python Crash Course" along with their corresponding code examples.

\subsection*{1. print() - Output Function}
\textbf{Definition}: A built-in Python function used to display text or values on the screen.

\lstinputlisting[language=Python, caption=Hello World program]{Chapter01/101_hello_world.py}

\subsection*{2. String Literal - Direct Text Value}
\textbf{Definition}: Text enclosed in quotes that represents a string value directly in the code.

\begin{lstlisting}
"Hello Python world!"
\end{lstlisting}

\subsection*{3. Function Call - Executing Code}
\textbf{Definition}: The process of running a function by using its name followed by parentheses.

\begin{lstlisting}
print("Hello Python world!")
\end{lstlisting}

\subsection*{4. String - Text Data Type}
\textbf{Definition}: A series of characters surrounded by single or double quotes. Strings are used to represent text in Python.

\begin{lstlisting}
"Hello Python world!"
\end{lstlisting}

\subsection*{5. Comment - Code Documentation}
\textbf{Definition}: Text in code that is ignored by Python but provides information to programmers about what the code does.

\begin{lstlisting}
# This is a comment explaining the code
print("Hello world!")  # This comment is on the same line
\end{lstlisting}

\subsection*{6. Syntax - Programming Rules}
\textbf{Definition}: The set of rules that define how Python code must be written to be understood by the interpreter.

\subsection*{7. Interpreter - Python Engine}
\textbf{Definition}: The program that reads and executes Python code line by line.

\subsection*{8. Whitespace - Spaces and Formatting}
\textbf{Definition}: Spaces, tabs, and newlines that affect how code is formatted and sometimes how it runs.

\subsection*{9. Error Message - Debugging Information}
\textbf{Definition}: Text displayed when Python encounters an error, helping programmers identify and fix problems.

\section*{Practical Examples from Chapter 1}

\subsection*{Running Your First Python Program}
The file \texttt{Chapter01/101\_hello\_world.py} contains your first Python program:

\lstinputlisting[language=Python, caption=Chapter01/101\_hello\_world.py]{Chapter01/101_hello_world.py}

\textbf{To run this program:}
\begin{verbatim}
python Chapter01/101_hello_world.py
\end{verbatim}

\textbf{Expected output:}
\begin{verbatim}
Hello Python world!
\end{verbatim}

\section*{Summary}
These keywords form the foundation of Python programming and are essential concepts that every beginner needs to understand. Chapter 1 focuses on the most basic concepts: using the print() function, working with strings, and understanding Python syntax.

\section*{Key Takeaways}
\begin{itemize}
    \item \texttt{print()} is the most basic way to output information
    \item Strings are the primary way to work with text in Python
    \item Comments help make code readable and maintainable
    \item Python is case-sensitive and follows specific syntax rules
    \item Always test your code by running it to see the output
    \item The interpreter reads and executes your code line by line
\end{itemize} 
