% Chapter 6 Content - Dictionaries
% This file contains the content for Chapter 6 based on actual Python files

This document contains the essential keywords and definitions from Chapter 6 of "Python Crash Course" along with their corresponding code examples.

\subsection*{1. Dictionary - Key-Value Pairs}
\textbf{Definition}: A collection of key-value pairs that allows you to connect pieces of related information.

\lstinputlisting[language=Python, caption=Basic dictionary operations]{Chapter06/601_alien.py}

\subsection*{2. Key-Value Pair - Dictionary Element}
\textbf{Definition}: A set of values associated with each other, where a key is used to access its associated value.

\begin{lstlisting}
alien_0 = {'color': 'green', 'points': 5}
\end{lstlisting}

\subsection*{3. Accessing Values - Dictionary Lookup}
\textbf{Definition}: The process of retrieving a value from a dictionary using its key.

\begin{lstlisting}
alien_0 = {'color': 'green', 'points': 5}
print(alien_0['color'])  # 'green'
\end{lstlisting}

\subsection*{4. Adding Key-Value Pairs - Dictionary Modification}
\textbf{Definition}: The process of adding new key-value pairs to an existing dictionary.

\begin{lstlisting}
alien_0 = {'color': 'green', 'points': 5}
alien_0['x_position'] = 0
alien_0['y_position'] = 25
\end{lstlisting}

\subsection*{5. Starting with Empty Dictionary - Dynamic Creation}
\textbf{Definition}: Creating a dictionary with no key-value pairs and adding them as needed.

\begin{lstlisting}
alien_0 = {}
alien_0['color'] = 'green'
alien_0['points'] = 5
\end{lstlisting}

\subsection*{6. Modifying Values - Dictionary Updates}
\textbf{Definition}: Changing the value associated with a key in a dictionary.

\begin{lstlisting}
alien_0 = {'color': 'green', 'points': 5}
alien_0['color'] = 'yellow'
\end{lstlisting}

\subsection*{7. Removing Key-Value Pairs - del Statement}
\textbf{Definition}: Permanently removing a key-value pair from a dictionary using the del statement.

\begin{lstlisting}
alien_0 = {'color': 'green', 'points': 5}
del alien_0['points']
\end{lstlisting}

\subsection*{8. Looping Through Dictionary - items() Method}
\textbf{Definition}: Iterating through all key-value pairs in a dictionary.

\lstinputlisting[language=Python, caption=Looping through dictionaries]{Chapter06/602_favourite_languages.py}

\subsection*{9. Looping Through Keys - keys() Method}
\textbf{Definition}: Iterating through all keys in a dictionary.

\begin{lstlisting}
favourite_languages = {'jen': 'python', 'sarah': 'c'}
for name in favourite_languages.keys():
    print(name.title())
\end{lstlisting}

\subsection*{10. Looping Through Values - values() Method}
\textbf{Definition}: Iterating through all values in a dictionary.

\begin{lstlisting}
favourite_languages = {'jen': 'python', 'sarah': 'c'}
for language in favourite_languages.values():
    print(language.title())
\end{lstlisting}

\subsection*{11. Nesting - Dictionaries in Dictionaries}
\textbf{Definition}: Storing multiple dictionaries in a list, or a list of items as a value in a dictionary.

\begin{lstlisting}
aliens = []
for alien_number in range(30):
    new_alien = {'color': 'green', 'points': 5, 'speed': 'slow'}
    aliens.append(new_alien)
\end{lstlisting}

\subsection*{12. List in Dictionary - Complex Data}
\textbf{Definition}: Using a list as a value in a dictionary to store multiple items.

\begin{lstlisting}
favourite_languages = {
    'jen': ['python', 'ruby'],
    'sarah': ['c'],
    'edward': ['ruby', 'go']
}
\end{lstlisting}

\subsection*{13. Dictionary in Dictionary - Nested Structures}
\textbf{Definition}: Storing a dictionary as a value in another dictionary.

\begin{lstlisting}
users = {
    'aeinstein': {
        'first': 'albert',
        'last': 'einstein',
        'location': 'princeton'
    }
}
\end{lstlisting}

\section*{Practical Examples from Chapter 6}

\subsection*{Working with Dictionaries}
Chapter 6 introduces dictionaries and their operations. Here are the key files:

\textbf{Basic Dictionary Operations:}
\lstinputlisting[language=Python, caption=Chapter06/601\_alien.py]{Chapter06/601_alien.py}

\textbf{Advanced Dictionary Operations:}
\lstinputlisting[language=Python, caption=Chapter06/602\_favourite\_languages.py]{Chapter06/602_favourite_languages.py}

\textbf{To run these programs:}
\begin{verbatim}
python Chapter06/601_alien.py
python Chapter06/602_favourite_languages.py
\end{verbatim}

\section*{Summary}
Chapter 6 focuses on dictionaries, which are collections of key-value pairs. You learn how to store and organize information in dictionaries, access and modify their contents, and loop through their data.

\section*{Key Takeaways}
\begin{itemize}
    \item Dictionaries store key-value pairs
    \item Use square brackets to access values by key
    \item Add new key-value pairs by assigning to a new key
    \item Modify values by assigning to an existing key
    \item Use del to remove key-value pairs
    \item Loop through all key-value pairs with .items()
    \item Loop through keys with .keys() (default behavior)
    \item Loop through values with .values()
    \item Use set() to get unique values
    \item Dictionaries can store lists and other dictionaries
    \item Nesting allows complex data structures
\end{itemize} 