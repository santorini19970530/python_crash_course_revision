% Chapter 10 Content - Files and Exceptions
% This file contains the content for Chapter 10 based on actual Python files

This document contains the essential keywords and definitions from Chapter 10 of "Python Crash Course" along with their corresponding code examples.

\subsection*{1. File - Data Storage}
\textbf{Definition}: A collection of information stored as a unit on a computer, accessible by programs.

\begin{lstlisting}
filename = 'pi_digits.txt'
with open(filename) as file_object:
    contents = file_object.read()
\end{lstlisting}

\subsection*{2. open() Function - File Access}
\textbf{Definition}: A function that opens a file and returns a file object, which contains methods and attributes for working with the file.

\begin{lstlisting}
with open('pi_digits.txt') as file_object:
    contents = file_object.read()
\end{lstlisting}

\subsection*{3. with Statement - File Context}
\textbf{Definition}: A statement that ensures a file is properly closed after the block of code using it is finished.

\begin{lstlisting}
with open('pi_digits.txt') as file_object:
    contents = file_object.read()
# File is automatically closed here
\end{lstlisting}

\subsection*{4. read() Method - File Content}
\textbf{Definition}: A method that reads the entire contents of a file as a string.

\lstinputlisting[language=Python, caption=Reading file contents]{Chapter10/1001_pi/file_reader.py}

\subsection*{5. readlines() Method - Line List}
\textbf{Definition}: A method that reads each line from a file and stores them in a list.

\begin{lstlisting}
with open('pi_digits.txt') as file_object:
    lines = file_object.readlines()

for line in lines:
    print(line.rstrip())
\end{lstlisting}

\subsection*{6. write() Method - File Writing}
\textbf{Definition}: A method that writes a string to a file, overwriting the file's contents.

\begin{lstlisting}
filename = 'programming.txt'
with open(filename, 'w') as file_object:
    file_object.write("I love programming.")
\end{lstlisting}

\subsection*{7. append Mode - 'a' Parameter}
\textbf{Definition}: A file mode that adds content to the end of a file instead of overwriting it.

\begin{lstlisting}
filename = 'programming.txt'
with open(filename, 'a') as file_object:
    file_object.write("I also love finding meaning in large datasets.\n")
\end{lstlisting}

\subsection*{8. Exception - Error Handling}
\textbf{Definition}: An error that occurs during program execution, which can be caught and handled.

\lstinputlisting[language=Python, caption=Exception handling]{Chapter10/1004_division_calculator.py}

\subsection*{9. try-except Block - Error Catching}
\textbf{Definition}: A block of code that tries to run some code and catches any exceptions that occur.

\begin{lstlisting}
try:
    answer = int(first_number) / int(second_number)
except ZeroDivisionError:
    print("You can't divide by 0!")
\end{lstlisting}

\subsection*{10. else Block - Success Handling}
\textbf{Definition}: A block of code that runs only if the try block succeeds (no exceptions occur).

\begin{lstlisting}
try:
    answer = int(first_number) / int(second_number)
except ZeroDivisionError:
    print("You can't divide by 0!")
else:
    print(answer)
\end{lstlisting}

\subsection*{11. FileNotFoundError - Missing File}
\textbf{Definition}: An exception that occurs when trying to open a file that doesn't exist.

\lstinputlisting[language=Python, caption=Handling missing files]{Chapter10/1006_word_count.py}

\subsection*{12. ZeroDivisionError - Division by Zero}
\textbf{Definition}: An exception that occurs when trying to divide by zero.

\begin{lstlisting}
try:
    print(5/0)
except ZeroDivisionError:
    print("You can't divide by zero!")
\end{lstlisting}

\subsection*{13. ValueError - Invalid Conversion}
\textbf{Definition}: An exception that occurs when trying to convert a string to a number when the string doesn't contain a valid number.

\begin{lstlisting}
try:
    age = int(input("Enter your age: "))
except ValueError:
    print("Please enter a valid number.")
\end{lstlisting}

\subsection*{14. pass Statement - Silent Failure}
\textbf{Definition}: A statement that tells Python to do nothing in a block, often used in exception handling.

\begin{lstlisting}
try:
    with open(filename) as f:
        contents = f.read()
except FileNotFoundError:
    pass  # Do nothing if file not found
\end{lstlisting}

\subsection*{15. JSON - Data Format}
\textbf{Definition}: A lightweight data format that's easy for programs to parse and generate.

\begin{lstlisting}
import json

numbers = [2, 3, 5, 7, 11, 13]
filename = 'numbers.json'
with open(filename, 'w') as f:
    json.dump(numbers, f)
\end{lstlisting}

\section*{Practical Examples from Chapter 10}

\subsection*{Working with Files and Exceptions}
Chapter 10 introduces file handling and exception handling. Here are the key files:

\textbf{File Reading Operations:}
\lstinputlisting[language=Python, caption=Chapter10/1001\_pi/file\_reader.py]{Chapter10/1001_pi/file_reader.py}

\textbf{Exception Handling:}
\lstinputlisting[language=Python, caption=Chapter10/1004\_division\_calculator.py]{Chapter10/1004_division_calculator.py}

\textbf{File Error Handling:}
\lstinputlisting[language=Python, caption=Chapter10/1006\_word\_count.py]{Chapter10/1006_word_count.py}

\textbf{To run these programs:}
\begin{verbatim}
python Chapter10/1001_pi/file_reader.py
python Chapter10/1004_division_calculator.py
python Chapter10/1006_word_count.py
\end{verbatim}

\section*{Summary}
Chapter 10 focuses on files and exceptions. You learn how to read from and write to files, handle errors gracefully, and work with different file formats like JSON.

\section*{Key Takeaways}
\begin{itemize}
    \item Files store data persistently
    \item open() creates file objects
    \item with ensures proper file closing
    \item read() gets entire file content
    \item readlines() gets list of lines
    \item write() overwrites file content
    \item 'a' mode appends to files
    \item Exceptions handle errors gracefully
    \item try-except catches exceptions
    \item else runs on successful try
    \item FileNotFoundError for missing files
    \item ZeroDivisionError for division by zero
    \item ValueError for invalid conversions
    \item pass does nothing in a block
    \item JSON stores structured data
\end{itemize} 