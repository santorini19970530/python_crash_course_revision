% Chapter 2 Content - Variables and Simple Data Types
% This file contains the content for Chapter 2 based on actual Python files

This document contains the essential keywords and definitions from Chapter 2 of "Python Crash Course" along with their corresponding code examples.

\subsection*{1. Variable - Storage Container}
\textbf{Definition}: A name that represents a value stored in memory. Variables are used to store and reference data.

\lstinputlisting[language=Python, caption=Variables and strings]{Chapter02/201_hello_world.py}

\subsection*{2. String Methods - Text Operations}
\textbf{Definition}: Built-in functions that can be called on strings to modify or analyze them.

\lstinputlisting[language=Python, caption=String methods]{Chapter02/202_name.py}

\subsection*{3. f-string - Formatted String}
\textbf{Definition}: A string that contains variables or expressions inside curly braces \{\}, prefixed with 'f'. Allows embedding variables directly in strings.

\lstinputlisting[language=Python, caption=f-strings and string formatting]{Chapter02/203_full_name.py}

\subsection*{4. String Concatenation - Combining Text}
\textbf{Definition}: The process of joining strings together using the + operator or f-strings.

\begin{lstlisting}
first_name = "ada"
last_name = "lovelace"
full_name = first_name + " " + last_name
\end{lstlisting}

\subsection*{5. Variable Assignment - Storing Values}
\textbf{Definition}: The process of storing a value in a variable using the = operator.

\begin{lstlisting}
name = "ada lovelace"
message = "Hello, " + name.title()
\end{lstlisting}

\subsection*{6. String Methods - title(), upper(), lower()}
\textbf{Definition}: Methods that modify the case of strings.

\begin{lstlisting}
name = "ada lovelace"
print(name.title())  # Ada Lovelace
print(name.upper())  # ADA LOVELACE
print(name.lower())  # ada lovelace
\end{lstlisting}

\subsection*{7. Comments - Code Documentation}
\textbf{Definition}: Text in code that is ignored by Python but provides information to programmers about what the code does.

\begin{lstlisting}
# This is a comment explaining the code
name = "ada"  # This comment is on the same line
\end{lstlisting}

\subsection*{8. String Formatting - .format() Method}
\textbf{Definition}: An older method of formatting strings by using placeholders and the .format() method.

\begin{lstlisting}
first_name = "ada"
last_name = "lovelace"
full_name = "{} {}".format(first_name, last_name)
\end{lstlisting}

\section*{Practical Examples from Chapter 2}

\subsection*{Working with Variables and Strings}
Chapter 2 introduces variables and string manipulation. Here are the key files:

\textbf{Basic Variables:}
\lstinputlisting[language=Python, caption=Chapter02/201\_hello\_world.py]{Chapter02/201_hello_world.py}

\textbf{String Methods:}
\lstinputlisting[language=Python, caption=Chapter02/202\_name.py]{Chapter02/202_name.py}

\textbf{f-strings and Formatting:}
\lstinputlisting[language=Python, caption=Chapter02/203\_full\_name.py]{Chapter02/203_full_name.py}

\textbf{To run these programs:}
\begin{verbatim}
python Chapter02/201_hello_world.py
python Chapter02/202_name.py
python Chapter02/203_full_name.py
\end{verbatim}

\section*{Summary}
Chapter 2 focuses on variables, string manipulation, and different ways to format strings. You learn how to store data in variables and use various string methods to modify text.

\section*{Key Takeaways}
\begin{itemize}
    \item Variables store data that can be reused throughout a program
    \item Strings are the primary way to work with text in Python
    \item f-strings provide a convenient way to embed variables in text
    \item String methods like title(), upper(), and lower() modify text case
    \item Comments help make code readable and maintainable
    \item The .format() method is an alternative to f-strings
    \item Variable names should be descriptive and follow naming conventions
\end{itemize} 