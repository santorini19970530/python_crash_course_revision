% Chapter 11 Content - Testing Your Code
% This file contains the content for Chapter 11 based on actual Python files

This document contains the essential keywords and definitions from Chapter 11 of "Python Crash Course" along with their corresponding code examples.

\subsection*{1. Test - Code Verification}
\textbf{Definition}: A piece of code that verifies that another piece of code works correctly.

\begin{lstlisting}
def test_first_last_name():
    """Do names like 'Janis Joplin' work?"""
    formatted_name = get_formatted_name('janis', 'joplin')
    assert formatted_name == 'Janis Joplin'
\end{lstlisting}

\subsection*{2. Unit Test - Function Testing}
\textbf{Definition}: A test that verifies that one aspect of a function works correctly.

\lstinputlisting[language=Python, caption=Unit testing with unittest]{Chapter11/test_name_function.py}

\subsection*{3. Test Case - Test Class}
\textbf{Definition}: A class that contains a series of unit tests that can be run together.

\begin{lstlisting}
class NamesTestCase(unittest.TestCase):
    """Tests for 'name_function.py'."""
    
    def test_first_last_name(self):
        """Do names like 'Janis Joplin' work?"""
        formatted_name = get_formatted_name('janis', 'joplin')
        self.assertEqual(formatted_name, 'Janis Joplin')
\end{lstlisting}

\subsection*{4. assertEqual() Method - Value Comparison}
\textbf{Definition}: A method that verifies that a value you expect matches the value the function returns.

\begin{lstlisting}
formatted_name = get_formatted_name('janis', 'joplin')
self.assertEqual(formatted_name, 'Janis Joplin')
\end{lstlisting}

\subsection*{5. unittest Module - Testing Framework}
\textbf{Definition}: A Python module that provides tools for testing your code.

\begin{lstlisting}
import unittest
from name_function import get_formatted_name

class NamesTestCase(unittest.TestCase):
    def test_first_last_name(self):
        formatted_name = get_formatted_name('janis', 'joplin')
        self.assertEqual(formatted_name, 'Janis Joplin')
\end{lstlisting}

\subsection*{6. setUp() Method - Test Preparation}
\textbf{Definition}: A method that runs before each test method, allowing you to create objects once and use them in all your test methods.

\begin{lstlisting}
def setUp(self):
    """Create a survey and a set of responses for use in all test methods."""
    question = "What language did you first learn to speak?"
    self.my_survey = AnonymousSurvey(question)
    self.responses = ['English', 'Spanish', 'Mandarin']
\end{lstlisting}

\subsection*{7. Test Function - Individual Test}
\textbf{Definition}: A function that tests a specific aspect of your code.

\lstinputlisting[language=Python, caption=Function to be tested]{Chapter11/name_function.py}

\subsection*{8. assertIn() Method - Membership Test}
\textbf{Definition}: A method that verifies that an item is in a list.

\begin{lstlisting}
def test_store_single_response(self):
    """Test that a single response is stored properly."""
    self.my_survey.store_response(self.responses[0])
    self.assertIn(self.responses[0], self.my_survey.responses)
\end{lstlisting}

\subsection*{9. assertNotIn() Method - Non-Membership Test}
\textbf{Definition}: A method that verifies that an item is not in a list.

\begin{lstlisting}
def test_duplicate_responses(self):
    """Test that duplicate responses are not stored."""
    self.my_survey.store_response(self.responses[0])
    self.my_survey.store_response(self.responses[0])
    self.assertEqual(len(self.my_survey.responses), 1)
\end{lstlisting}

\subsection*{10. Test Runner - Test Execution}
\textbf{Definition}: A tool that runs your tests and reports the results.

\begin{lstlisting}
if __name__ == '__main__':
    unittest.main()
\end{lstlisting}

\subsection*{11. Failing Test - Bug Detection}
\textbf{Definition}: A test that fails, indicating that there's a problem with the code being tested.

\begin{lstlisting}
def test_first_last_middle_name(self):
    """Do names like 'Wolfgang Amadeus Mozart' work?"""
    formatted_name = get_formatted_name('wolfgang', 'mozart', 'amadeus')
    self.assertEqual(formatted_name, 'Wolfgang Amadeus Mozart')
\end{lstlisting}

\subsection*{12. Passing Test - Success Verification}
\textbf{Definition}: A test that passes, indicating that the code being tested works correctly.

\begin{lstlisting}
def test_first_last_name(self):
    """Do names like 'Janis Joplin' work?"""
    formatted_name = get_formatted_name('janis', 'joplin')
    self.assertEqual(formatted_name, 'Janis Joplin')
\end{lstlisting}

\subsection*{13. Test Coverage - Code Verification}
\textbf{Definition}: The percentage of your code that's covered by tests.

\begin{lstlisting}
# Test different scenarios
def test_empty_string(self):
    """Test with empty strings."""
    result = get_formatted_name('', '')
    self.assertEqual(result, ' ')

def test_single_name(self):
    """Test with single name."""
    result = get_formatted_name('john', '')
    self.assertEqual(result, 'John ')
\end{lstlisting}

\subsection*{14. Integration Test - System Testing}
\textbf{Definition}: A test that verifies that multiple parts of your system work together correctly.

\lstinputlisting[language=Python, caption=Integration testing with user input]{Chapter11/language_survey.py}

\section*{Practical Examples from Chapter 11}

\subsection*{Working with Testing}
Chapter 11 introduces testing and test-driven development. Here are the key files:

\textbf{Function to be Tested:}
\lstinputlisting[language=Python, caption=Chapter11/name\_function.py]{Chapter11/name_function.py}

\textbf{Unit Tests:}
\lstinputlisting[language=Python, caption=Chapter11/test\_name\_function.py]{Chapter11/test_name_function.py}

\textbf{Integration Testing:}
\lstinputlisting[language=Python, caption=Chapter11/language\_survey.py]{Chapter11/language_survey.py}

\textbf{To run these tests:}
\begin{verbatim}
python Chapter11/test_name_function.py
python Chapter11/language_survey.py
\end{verbatim}

\section*{Summary}
Chapter 11 focuses on testing your code. You learn how to write tests that verify your functions work correctly, how to test for different scenarios, and how to use Python's unittest framework.

\section*{Key Takeaways}
\begin{itemize}
    \item Tests verify code works correctly
    \item Unit tests check individual functions
    \item Test cases group related tests
    \item assertEqual() compares expected and actual values
    \item unittest provides testing framework
    \item setUp() prepares test data
    \item assertIn() checks list membership
    \item assertNotIn() checks non-membership
    \item Test runners execute tests
    \item Failing tests indicate bugs
    \item Passing tests verify correctness
    \item Test coverage measures completeness
    \item Integration tests check system parts
    \item Write tests before fixing bugs
    \item Tests help prevent regressions
\end{itemize} 