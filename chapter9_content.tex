% Chapter 9 Content - Classes
% This file contains the content for Chapter 9 based on actual Python files

This document contains the essential keywords and definitions from Chapter 9 of "Python Crash Course" along with their corresponding code examples.

\subsection*{1. Class - Object Blueprint}
\textbf{Definition}: A blueprint for creating objects, defining what attributes and methods the objects will have.

\lstinputlisting[language=Python, caption=Basic class definition]{Chapter09/901_dog.py}

\subsection*{2. Object - Class Instance}
\textbf{Definition}: An instance of a class that contains data and behavior defined by the class.

\begin{lstlisting}
my_dog = Dog('Willie', 6)
your_dog = Dog('Lucy', 3)
\end{lstlisting}

\subsection*{3. Attribute - Object Data}
\textbf{Definition}: A variable that belongs to an object, accessed using dot notation.

\begin{lstlisting}
print(f"My dog's name is {my_dog.name}.")
print(f"My dog is {my_dog.age} years old.")
\end{lstlisting}

\subsection*{4. Method - Object Behavior}
\textbf{Definition}: A function that belongs to a class, defining what an object can do.

\begin{lstlisting}
my_dog.sit()
my_dog.roll_over()
\end{lstlisting}

\subsection*{5. \_\_init\_\_() Method - Constructor}
\textbf{Definition}: A special method that Python runs automatically whenever you create a new instance of a class.

\begin{lstlisting}
def __init__(self, name, age):
    """Initialize name and age attributes."""
    self.name = name
    self.age = age
\end{lstlisting}

\subsection*{6. self Parameter - Object Reference}
\textbf{Definition}: A reference to the instance of the class, allowing you to access attributes and methods.

\begin{lstlisting}
def sit(self):
    """Simulate a dog sitting in response to a command."""
    print(f"{self.name} is now sitting.")
\end{lstlisting}

\subsection*{7. Instance - Class Object}
\textbf{Definition}: An individual object created from a class, with its own set of attributes.

\begin{lstlisting}
my_dog = Dog('Willie', 6)  # my_dog is an instance
your_dog = Dog('Lucy', 3)  # your_dog is another instance
\end{lstlisting}

\subsection*{8. Inheritance - Class Relationship}
\textbf{Definition}: A feature that allows you to model relationships between classes, where a child class inherits attributes and methods from a parent class.

\lstinputlisting[language=Python, caption=Inheritance example]{Chapter09/electric_car.py}

\subsection*{9. Parent Class - Base Class}
\textbf{Definition}: A class that is inherited from, also called a base class or superclass.

\begin{lstlisting}
class Car:
    """A simple attempt to represent a car."""
    def __init__(self, make, model, year):
        self.make = make
        self.model = model
        self.year = year
\end{lstlisting}

\subsection*{10. Child Class - Derived Class}
\textbf{Definition}: A class that inherits from another class, also called a derived class or subclass.

\begin{lstlisting}
class ElectricCar(Car):
    """Represents aspects of a car, specific to electric vehicles."""
    def __init__(self, make, model, year):
        super().__init__(make, model, year)
\end{lstlisting}

\subsection*{11. super() Function - Parent Access}
\textbf{Definition}: A function that helps you make connections between parent and child classes.

\begin{lstlisting}
def __init__(self, make, model, year):
    super().__init__(make, model, year)
    self.battery = Battery()
\end{lstlisting}

\subsection*{12. Method Overriding - Custom Behavior}
\textbf{Definition}: The ability to define a method in a child class that has the same name as a method in the parent class.

\begin{lstlisting}
def fill_gas_tank(self):
    """Electric cars don't have gas tanks."""
    print("This car doesn't need a gas tank!")
\end{lstlisting}

\subsection*{13. Instance as Attribute - Object Composition}
\textbf{Definition}: Using an instance of one class as an attribute in another class.

\begin{lstlisting}
class Battery:
    def __init__(self, battery_size=75):
        self.battery_size = battery_size

class ElectricCar(Car):
    def __init__(self, make, model, year):
        super().__init__(make, model, year)
        self.battery = Battery()  # Instance as attribute
\end{lstlisting}

\subsection*{14. Importing Classes - Module Usage}
\textbf{Definition}: Bringing classes from one module into another module for use.

\begin{lstlisting}
from car import Car
from electric_car import ElectricCar

my_tesla = ElectricCar('tesla', 'model s', 2019)
\end{lstlisting}

\section*{Practical Examples from Chapter 9}

\subsection*{Working with Classes}
Chapter 9 introduces object-oriented programming with classes. Here are the key files:

\textbf{Basic Class Definition:}
\lstinputlisting[language=Python, caption=Chapter09/901\_dog.py]{Chapter09/901_dog.py}

\textbf{Advanced Class with Methods:}
\lstinputlisting[language=Python, caption=Chapter09/902\_car.py]{Chapter09/902_car.py}

\textbf{Inheritance and Class Relationships:}
\lstinputlisting[language=Python, caption=Chapter09/electric\_car.py]{Chapter09/electric_car.py}

\textbf{To run these programs:}
\begin{verbatim}
python Chapter09/901_dog.py
python Chapter09/902_car.py
python Chapter09/electric_car.py
\end{verbatim}

\section*{Summary}
Chapter 9 focuses on classes and object-oriented programming. You learn how to create classes, define attributes and methods, create instances, and use inheritance to model relationships between classes.

\section*{Key Takeaways}
\begin{itemize}
    \item Classes are blueprints for creating objects
    \item Objects are instances of classes
    \item Attributes store data in objects
    \item Methods define behavior of objects
    \item \_\_init\_\_() initializes new instances
    \item self refers to the current instance
    \item Inheritance creates class relationships
    \item Child classes inherit from parent classes
    \item super() calls parent class methods
    \item Method overriding customizes behavior
    \item Objects can contain other objects
    \item Import classes to use them in other modules
    \item Classes help organize and structure code
\end{itemize} 